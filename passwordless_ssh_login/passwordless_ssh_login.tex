\documentclass[12pt]{article}
\usepackage[margin=.75in]{geometry}
\usepackage{titling}
\usepackage[sc]{mathpazo}
\linespread{1.05}         % Palatino needs more leading (space between lines)
\usepackage[T1]{fontenc}
\usepackage{fancyvrb}

\setlength{\droptitle}{-5em}

\title{Passwordless SSH Login}
\author{Mark Froilan B. Tandoc}
\date{24 February 2015}

\begin{document}

\maketitle

\section*{Description}
Allow ssh login to a remote host without the need to input password.

\section*{Definitions}
\begin{itemize}
\item \textbf{remote-host}: machine where to login
\item \textbf{local-host}: machine on site
\item \textbf{user}: current user
\end{itemize}

\section*{Method}
\begin{itemize}
\item Create public and private keys using ssh-keygen on local host

\begin{Verbatim}[frame=single]
user@local-host$ ssh-keygen
\end{Verbatim}

\item Press enter when asked where to save the key to save it in default directory and use an empty passphrase (Just press enter). Sample output is shown below.

\begin{Verbatim}[frame=single]
Generating public/private rsa key pair.
Enter file in which to save the key (/home/user/.ssh/id_rsa): 
Enter passphrase (empty for no passphrase): 
Enter same passphrase again: 
Your identification has been saved in /home/user/.ssh/id_rsa.
Your public key has been saved in /home/user/.ssh/id_rsa.pub.
The key fingerprint is:
49:95:7e:26:33:e4:03:51:6a:e3:4d:f8:75:03:e7:07 user@local-host
The key's randomart image is:
\end{Verbatim}

\item Copy the public key to remote-host using ssh-copy-id

\begin{Verbatim}[frame=single]
user@local-host:~$ ssh-copy-id -i user@remote-host
/usr/bin/ssh-copy-id: INFO: attempting to log in with the new key(s), to
filter out any that are already installed
/usr/bin/ssh-copy-id: INFO: 1 key(s) remain to be installed -- if you are
prompted now it is to install the new keys
user@remote-host's password:

Number of key(s) added: 1

Now try logging into the machine, with:   "ssh 'user@remote-host'"
and check to make sure that only the key(s) you wanted were added.
\end{Verbatim}

\item ssh-copy-id appends the keys to the remote-host's .ssh/authorized\_key.

\end{itemize}

\section*{Reference}
\begin{itemize}
\item http://www.thegeekstuff.com/2008/11/3-steps-to-perform-ssh-login-without-password-using-ssh-keygen-ssh-copy-id/
\end{itemize}

\end{document}